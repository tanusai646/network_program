\documentclass[documentclass]{jsarticle}
\usepackage[top=25truemm,bottom=25truemm,left=20truemm,right=20truemm]{geometry}
\usepackage{listings, jlisting, color}
\usepackage[dvipdfmx]{graphicx}
\usepackage{pdfpages}
\usepackage{amsmath}
\usepackage{amssymb, latexsym}
\usepackage{mathtools}
\usepackage{multirow}
\usepackage{color}
\usepackage{ulem}
\usepackage{here}
\usepackage{wrapfig}
\usepackage{tikz}
% 使用する関数の宣言
\usepackage{url}
\usepackage{hyperref}
\hypersetup{
    pdfauthor={222C1021 今村優希},
}
\usetikzlibrary{intersections, calc, arrows, positioning, arrows.meta}

\newcommand{\Add}[1]{\textcolor{red}{#1}}
\newcommand{\Erase}[1]{\textcolor{red}{\sout{\textcolor{black}{#1}}}}
\newcommand{\ctext}[1]{\raise0.2ex\hbox{\textcircled{\scriptsize{#1}}}}

\lstset{
  basicstyle=\ttfamily,
  breaklines=true,
  frame=single,
  tabsize=3,
}

\begin{document}
\title{ネットワークプログラミング\\ 第10回演習レポート}
\author{222C1021 今村優希}
\maketitle

\newpage

\section*{演習2}
作成したものはprint.cとprint.htmlである.
これらプログラムを実行した際の画像が図\ref*{fig:2-1}から\ref*{fig:2-3}である.
htmlファイルにおいては,上段はGETを用いた送信,下段はPOSTを用いた送信を可能にしている.
出力に関しては,\texttt{original string}の下に文字列を編集したものが出力されている.

\lstinputlisting[caption={print.c}]{code/print.c}
\lstinputlisting[caption={print.html}]{code/print.html}

\begin{figure}[H]
  \begin{center}
    \includegraphics*[scale=0.8]{figure/2-0.png}
  \end{center}
  \caption[]{htmlファイルの表示画面}
  \label{fig:2-1}
\end{figure}

\begin{figure}[H]
  \begin{center}
    \includegraphics*[scale=0.8]{figure/2-1.png}
  \end{center}
  \caption[]{GETで送信したときの結果}
  \label{fig:2-2}
\end{figure}

\begin{figure}[H]
  \begin{center}
    \includegraphics*[scale=0.9]{figure/2-2.png}
  \end{center}
  \caption[]{POSTで送信したときの結果}
  \label{fig:2-3}
\end{figure}
\newpage

\section*{演習3}
\texttt{利用なし}

\section*{備考}
\url{https://github.com/tanusai646/network_program/tree/main/6}

\end{document}