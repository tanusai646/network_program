\documentclass[documentclass]{jsarticle}
\usepackage[top=25truemm,bottom=25truemm,left=20truemm,right=20truemm]{geometry}
\usepackage{listings, jlisting, color}
\usepackage[dvipdfmx]{graphicx}
\usepackage{pdfpages}
\usepackage{amsmath}
\usepackage{amssymb, latexsym}
\usepackage{mathtools}
\usepackage{multirow}
\usepackage{color}
\usepackage{ulem}
\usepackage{here}
\usepackage{wrapfig}
\usepackage{tikz}
\usepackage{tcolorbox}
\usepackage{url}
\usepackage{hyperref}
\hypersetup{
    pdfauthor={222C1021 今村優希},
}
\usetikzlibrary{intersections, calc, arrows, positioning, arrows.meta}

\newcommand{\Add}[1]{\textcolor{red}{#1}}
\newcommand{\Erase}[1]{\textcolor{red}{\sout{\textcolor{black}{#1}}}}
\newcommand{\ctext}[1]{\raise0.2ex\hbox{\textcircled{\scriptsize{#1}}}}

\lstset{
  basicstyle=\ttfamily,
  breaklines=true,
  frame=single,
  tabsize=3,
}

\begin{document}
\title{ネットワークプログラミング\\ 最終レポート}
\author{222C1021 今村優希}
\maketitle

\newpage

\section*{演習2 WSによる簡易チャット/LINEの作成}
作成したものは\texttt{chat.html},\texttt{chat.js},\texttt{chat.css}で,\texttt{server.js}も多少変更を行った.
\texttt{chat.html}(ソースコード\ref*{sc:2-1})に関しては名前とメッセージを入力できるようした.
そして,\texttt{chat.js}(ソースコード\ref*{sc:2-2})によってその名前とメッセージの入力を確認し,ウェブソケットで送信し,受信した内容を表示をすることができる.
\texttt{chat.css}(ソースコード\ref*{sc:2-3})は,チャットのように表示させたかったため使用していたが,表示は中途半端になってしまった.

\texttt{server.js}(ソースコード\ref*{sc:2-4})に関しては,受信・送信するデータが違ったため,変更を加えた.

上記のhtmlを実行した結果が図\ref*{fig:2-1}で,\texttt{server.js}の動作結果が図\ref*{fig:2-2}である.
htmlの画面において,先に左側で名前とメッセージを入力し,送信をした.
次に,右側から送信を行った.
コンソールには2つのメッセージの内容が表示されているのが確認でき,また自分が入力したものは緑色で表示され,相手が入力してきたものは白色で表示されるようCSSを編集している.

\lstinputlisting[caption={chat.html}\label{sc:2-1}]{code/chat.html}
\lstinputlisting[caption={chat.js}\label{sc:2-2}]{code/chat.js}
\lstinputlisting[caption={chat.css}\label{sc:2-3}]{code/chat.css}

\lstinputlisting[caption={server.js}\label{sc:2-4}]{code/server.js}

%htmlの実行例
\begin{figure}[H]
  \begin{center}
    \includegraphics*[scale=0.35]{figure/2-1.png}
  \end{center}
  \caption[]{\texttt{chat.html}に接続し,メッセージの送信をした結果}
  \label{fig:2-1}
\end{figure}

%server動作の図
\begin{figure}[H]
  \begin{center}
    \includegraphics*[scale=0.7]{figure/2-2.png}
  \end{center}
  \caption[]{\texttt{server.js}の動作結果}
  \label{fig:2-2}
\end{figure}

\newpage

\section*{演習3 ソケット通信を利用した簡易FTPの実現}
\subsection*{3.1 ファイルを受診するクライアントプログラムの作成}
ソースコード\ref*{sc:3-1}は,ソケット通信を利用し,学生番号を入力し,受診した文字列をテキストファイルに格納するプログラムである.
図\ref*{fig:3-1}が,このプログラムを実行した際の出力である.

IPアドレス131.206.37.50に接続できた.
\texttt{222C1021.txt}の内容を受診し,\texttt{222C1021-copy.txt}に書き込むことができた.

\lstinputlisting[caption={ftp.c}\label{sc:3-1}]{code/ftp.c}
%ftp動作の図
\begin{figure}[H]
  \begin{center}
    \includegraphics*[]{figure/3-1.png}
  \end{center}
  \caption[]{\texttt{ftp.c}を実行した結果}
  \label{fig:3-1}
\end{figure}

\section*{3.2 取得ファイルの内容の確認,調査/報告}
取得したテキストファイルは以下の内容であった.
\begin{tcolorbox}
  課題:ルータにおけるIPデータグラム(パケット)転送

  1. インターネットはパケット単位の蓄積交換方式(パケット交換方式)を採用するが、その利点と問題点を説明せよ。
  
  2. 異種ネットワークを相互接続し、1.の主体となる「ルータ」において、到着パケットの転送先を決定する経路制御方式を一般的に何と呼ぶか答え、IPv4アドレスを仮定した場合、どのように実現するか調査、説明せよ。
  
  3. 代表的な経路情報交換プロトコルにRIP(Routing Information Protocol)とOSPF(Open Shortest Path First)がある。両者の違いを説明せよ。
\end{tcolorbox}

\subsubsection*{課題1}
まず,蓄積交換方式とは,参考文献[1]によると「インターネットではルータにおいてパケットをいったん蓄積して(store),次のルータに転送する(forward)を意味している」とある.

したがって,蓄積交換方式のメリットとして,複数人で利用する際にインターネット資源を共有することができるという点が挙げられる.
送信したいデータをパケットに分割することで,長いデータを送信したい等で資源を専有されることがなく,公平に使用することが可能である.

一方,デメリットとしてパケット到着の遅延と廃棄が発生する可能性があるという点が挙げられる.
通信回線の利用率が高いとき,複数のパケットがルータに到着する可能性があり,バッファで待機することで遅延が発生する.
さらに,この遅延が行き過ぎると,バッファにパケットが入らなくなり,パケットの廃棄が発生する.
そうなるとパケットの再送が必要で効率が落ちる可能性がある.

\subsubsection*{課題2}
ルータにおける到着パケットの転送先を決定する経路制御方式を\textbf{経路制御}または\textbf{ルーティング}という.
さらに,\textbf{ホップバイホップ制御}を使っている.
その制御の中でも2種類の制御方法があり,\textbf{静的経路制御}と\textbf{動的経路制御}が存在する.
静的経路制御は,ルータの管理者が経路表を自ら作成しているのに対して,動的経路制御は経路表をルータ自身で行っている.
よって,現在では後者の動的経路制御で行っているのが一般的である.

IPv4アドレスと仮定し動的経路制御を用いて転送を行うと考える.
まず,ルータは経路表を作成し,各宛先ネットワークの経路情報を保存している.
下記で詳しく解答する動的経路制御プロトコルを利用してルータ同士で経路情報を交換し,更新を行っている.
そして,ルータがパケットを受診したときに宛先IPアドレスを確認する.
そのIPアドレスと経路表を参照して次の転送先を決定し,転送を行っている.

\subsubsection*{課題3}
RIPは,ルータが自分のルーティングテーブルを隣接するルータと共有し合うことで経路情報を更新する距離ベクトル型経路制御プロトコルである.
各ルータは、宛先アドレスとその宛先に到達するためのメトリック(ホップ数)を交換している.
ホップ数とは、パケットが宛先に到達するまでに通過するルータの数を指している.
そのメトリックは定期的に隣接ルータに送信しており,その操作に時間がかかることがある.
受け取ったメトリックから,Bellman-Fordアルゴリズムを用いて経路表を作成し,最短経路で宛先アドレスに到達できるようにしている.
また,ホップ数の制限があるため,大規模なネットワークに向いておらず,小規模なネットワークで使用される.

OSPFはネットワーク内のルータが自分の接続(リンク)の状態に関する情報を交換し,ネットワークの最適な経路を計算するリンク状態型経路制御プロトコルである.
このプロトコルでは,ネットワーク内のすべてのルータおよびリンクに関する情報を収集し,それをもとにLSDBというリンク状態情報データベースを作成する.
このデータベースは帯域幅や遅延,その他のメトリックを使用しており,ホップ数のみを使用するRIPよりも正確なネットワークのマップを作成することが可能である.
そして,リンクの状態が変化したときのみアップデートを行うため,ネットワークのトラヒックが減る.
このような方法から,OSPFは大規模なネットワークで使用されており,今日のルータの多くで採用されている.



\subsection*{3.3 3.1の機能を提供するサーバープログラムの作成}
作成したものは,ftp-server.cとftp-client.cの2種類であり,ソースコード\ref*{sc:3-2},\ref*{sc:3-3}に示している.
また,これらのプログラムを実行した結果が図\ref*{fig:3-2}である.
サーバー側ではクライアントとと送受信を行う際に子プロセスを作成する.
クライアント側で学生番号を入力すると,サーバ側で学生番号を受け取り,学生番号.txtをオープンする.
サーバー側でも\texttt{exit\_send}が\texttt{buf}の中にあると読み込み,送信を終了する.
クライアントの動きは3.1と同じである.
図\ref*{fig:3-2}の状況からわかる通り,クライアントが終了したとしてもサーバーは受診待ちの状態である.

また,図\ref*{fig:3-3}は送信する内容(\texttt{222C1021.txt})で,図\ref*{fig:3-4}は受診した内容(\texttt{222C1021-test.c})である.

\lstinputlisting[caption={ftp-server.c}\label{sc:3-2}]{code/ftp-server.c}
\lstinputlisting[caption={ftp-client.c}\label{sc:3-3}]{code/ftp-client.c}

\begin{figure}[H]
  \begin{center}
    \includegraphics*[scale=0.7]{figure/3-2.png}
  \end{center}
  \caption{\texttt{ftp-server.c}と\texttt{ftp-client.c}の実行結果}
  \label{fig:3-2}
\end{figure}

\begin{figure}[H]
  \begin{center}
    \includegraphics*[]{figure/3-3.png}
  \end{center}
  \caption{送信元のデータ(\texttt{222C1021.txt})}
  \label{fig:3-3}
\end{figure}

\begin{figure}[H]
  \begin{center}
    \includegraphics*[]{figure/3-4.png}
  \end{center}
  \caption{受診したデータ(\texttt{222C1021-test.c})}
  \label{fig:3-4}
\end{figure}

\newpage
\section*{演習4 サーバー機能のクラウドへのデプロイ}

\newpage
\section*{備考}
\url{https://github.com/tanusai646/network_program/tree/main/7}

\begin{thebibliography}{9}
  \bibitem{1} 尾家祐二 後藤滋樹 小西和憲 西尾章治郎, 1 インターネット入門, 株式会社岩波書店, 2002 
  \bibitem{2} 堀良影 池永全志 門林雄基 後藤滋樹, 2 ネットワーク相互接続, 株式会社岩波書店, 2001
\end{thebibliography}

\end{document}