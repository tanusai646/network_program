\documentclass[documentclass]{jsarticle}
\usepackage[top=25truemm,bottom=25truemm,left=20truemm,right=20truemm]{geometry}
\usepackage{listings, jlisting, color}
\usepackage[dvipdfmx]{graphicx}
\usepackage{pdfpages}
\usepackage{amsmath}
\usepackage{amssymb, latexsym}
\usepackage{mathtools}
\usepackage{multirow}
\usepackage{color}
\usepackage{ulem}
\usepackage{here}
\usepackage{wrapfig}
\usepackage{tikz}
% 使用する関数の宣言
\usepackage{url}
% (最低限これさえ宣言していれば十分だと思われるものを書いています)
\usetikzlibrary{intersections, calc, arrows, positioning, arrows.meta}


\newcommand{\Add}[1]{\textcolor{red}{#1}}
\newcommand{\Erase}[1]{\textcolor{red}{\sout{\textcolor{black}{#1}}}}
\newcommand{\ctext}[1]{\raise0.2ex\hbox{\textcircled{\scriptsize{#1}}}}

\lstset{
  basicstyle=\ttfamily,
  breaklines=true,
  frame=single,
  tabsize=3,
}

\begin{document}
\title{ネットワークプログラミング\\ 第3回演習レポート}
\author{222C1021 今村優希}
\maketitle

\newpage

\section*{演習1}
左側のプログラムを実行させたときの結果は、図\ref*{1-1}である。
\begin{figure}[H]
  \begin{center}
    \includegraphics*[scale=0.8]{figure/1-1.png}
  \end{center}
  \caption{実行結果}
  \label{fig:1-1}
\end{figure}
また、右側のプログラムを実行させた。
"input1.dat"を読み込ませて、"output1.dat"に書き込ませ、出力させたものが図\ref*{fig:1-2}である。
\begin{figure}[H]
  \begin{center}
    \includegraphics*[scale=0.8]{figure/1-2.png}
  \end{center}
  \caption{output1.datの出力}
  \label{fig:1-2}
\end{figure}

\section*{演習2}
作成したのは、ソースコード1である。
このプログラムを実行した結果がである。
\lstinputlisting[caption={演習2のプログラム}]{code/e3-2.c}
\begin{figure}[H]
  \begin{center}
    \includegraphics*[scale=0.8]{figure/2-1.png}
  \end{center}
  \caption{ソースコード1の実行結果}
  \label{fig:2-1}
\end{figure}

\section*{演習3}
作成したプログラムは、ソースコード2である。
"output3.dat"に入力し、それを出力させたものが図\ref*{fig:3-1}である。
\lstinputlisting[caption={演習3のプログラム}]{code/e3-3.c}
\begin{figure}[H]
  \begin{center}
    \includegraphics*[scale=0.8]{figure/3-1.png}
  \end{center}
  \caption{"output3.dat"の出力結果}
  \label{fig:3-1}
\end{figure}

\section*{演習4}
プロセス生成を行った場合の実行結果が\ref*{fig:4-1}である。
また、リダイレクションを行った場合の実行結果が\ref*{fig:4-2}である。
リダイレクションは、temp.txtに出力が書き込まれている。
\begin{figure}[H]
  \begin{center}
    \includegraphics*[scale=0.8]{figure/4-1.png}
  \end{center}
  \caption{プロセス生成の実行結果}
  \label{fig:4-1}
\end{figure}
\begin{figure}[H]
  \begin{center}
    \includegraphics*[scale=0.8]{figure/4-2.png}
  \end{center}
  \caption{リダイレクションの実行結果}
  \label{fig:4-2}
\end{figure}

\section*{演習5}
まずは、p.17のコマンドライン実行を行った。
行った結果が、\ref*{fig:5-1}である。
\begin{figure}
  \begin{center}
    \includegraphics*[scale=0.8]{figure/5-1.png}
  \end{center}
  \caption{コマンドラインでの実行結果}
  \label{fig:5-1}
\end{figure}

また、親子プロセス間通信を実現するためのプログラムは、ソースコード3である。
このプログラムを実行させた時に結果が、\ref*{fig:5-2}である。
\lstinputlisting[caption={親子プロセス間を実現するプログラム}]{code/e3-5.c}
\begin{figure}[H]
  \begin{center}
    \includegraphics*[scale=0.8]{figure/5-2.png}
  \end{center}
  \caption{ソースコード3の実行結果}
  \label{fig:5-2}
\end{figure}

\end{document}