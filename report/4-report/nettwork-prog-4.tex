\documentclass[documentclass]{jsarticle}
\usepackage[top=25truemm,bottom=25truemm,left=20truemm,right=20truemm]{geometry}
\usepackage{listings, jlisting, color}
\usepackage[dvipdfmx]{graphicx}
\usepackage{pdfpages}
\usepackage{amsmath}
\usepackage{amssymb, latexsym}
\usepackage{mathtools}
\usepackage{multirow}
\usepackage{color}
\usepackage{ulem}
\usepackage{here}
\usepackage{wrapfig}
\usepackage{tikz}
% 使用する関数の宣言
\usepackage{url}
% (最低限これさえ宣言していれば十分だと思われるものを書いています)
\usetikzlibrary{intersections, calc, arrows, positioning, arrows.meta}


\newcommand{\Add}[1]{\textcolor{red}{#1}}
\newcommand{\Erase}[1]{\textcolor{red}{\sout{\textcolor{black}{#1}}}}
\newcommand{\ctext}[1]{\raise0.2ex\hbox{\textcircled{\scriptsize{#1}}}}

\lstset{
  basicstyle=\ttfamily,
  breaklines=true,
  frame=single,
  tabsize=3,
  numbers=left
}

\begin{document}
\title{ネットワークプログラミング\\ 第3回演習レポート}
\author{222C1021 今村優希}
\maketitle

\newpage

\section*{演習1}
左側のプログラムを実行させたときの結果は、ソースコード1である。
\lstinputlisting[caption={左側のプログラム実行結果}]{code/1-1.txt}
また、右側のプログラムを実行させた。
"input1.dat"を読み込ませて、"output1.dat"に書き込ませた結果がソースコード2,3である。
\lstinputlisting[caption={input.dat}]{code/1-2.dat}
\lstinputlisting[caption={output.dat}]{code/1-3.dat}

\section*{演習2}
作成したのは、ソースコード4である。
このプログラムを実行した結果がソースコード5である。
\lstinputlisting[caption={演習2のプログラム}]{code/2-1.c}
\begin{lstlisting}[caption={実行結果}]
  学生数 = 5
  平均点 = 83.000000
\end{lstlisting}

\section*{演習3}
作成したプログラムは、ソースコード6である。
その実行結果が、ソースコード7である。
\lstinputlisting[caption={演習3のプログラム}]{code/3-3.c}
\lstinputlisting[caption={output3.dat}]{code/3-4.dat}

\section*{演習4}
プロセス生成を行った場合の実行結果がソースコード8である。
また、リダイレクションを行った場合の実行結果がソースコード9である。
リダイレクションは、temp.txtに出力が書き込まれている。
\lstinputlisting[caption={プロセス生成の実行結果}]{code/4-1.txt}
\lstinputlisting[caption={temp.txt}]{code/4-2.txt}

\section*{演習5}
まずは、p.17のコマンドライン実行を行った。
行った結果が、ソースコード10である。
\lstinputlisting[caption={コマンドの実行結果}]{code/5-1.txt}

また、親子プロセス間通信を実現するためのプログラムは、ソースコード11である。
このプログラムを実行させた時に結果が、ソースコード12である。
\lstinputlisting[caption={親子プロセス間を実現するプログラム}]{code/5-2.c}
\lstinputlisting[caption={プログラムの実行結果}]{code/5-3.txt}
\end{document}