\documentclass[documentclass]{jsarticle}
\usepackage[top=25truemm,bottom=25truemm,left=20truemm,right=20truemm]{geometry}
\usepackage{listings, jlisting, color}
\usepackage[dvipdfmx]{graphicx}
\usepackage{pdfpages}
\usepackage{amsmath}
\usepackage{amssymb, latexsym}
\usepackage{mathtools}
\usepackage{multirow}
\usepackage{color}
\usepackage{ulem}
\usepackage{here}
\usepackage{wrapfig}
\usepackage{tikz}
% 使用する関数の宣言
\usepackage{url}
\usepackage{hyperref}
\hypersetup{
    pdfauthor={222C1021 今村優希},
}
\usetikzlibrary{intersections, calc, arrows, positioning, arrows.meta}

\newcommand{\Add}[1]{\textcolor{red}{#1}}
\newcommand{\Erase}[1]{\textcolor{red}{\sout{\textcolor{black}{#1}}}}
\newcommand{\ctext}[1]{\raise0.2ex\hbox{\textcircled{\scriptsize{#1}}}}

\lstset{
  basicstyle=\ttfamily,
  breaklines=true,
  frame=single,
  tabsize=3,
}

\begin{document}
\title{ネットワークプログラミング\\ 第6回演習レポート}
\author{222C1021 今村優希}
\maketitle

\newpage

\section*{演習1}
作成したserver.cとclient.cを実行した結果が図\ref*{fig:1-1}である.
server側で設定した文字列をclient側の出力できるのを確認できた.

%演習課題1の出力結果の図
\begin{figure}[H]
  \begin{center}
    \includegraphics*[scale=0.7]{figure/1-1.png}
  \end{center}
  \caption{server.cとclient.cの実行結果}
  \label{fig:1-1}
\end{figure}

\newpage
\section*{演習2}
作成したserver2.cとclient2.cはソースコード\ref*{cd:2-1},\ref*{cd:2-2}である.
このプログラムをコンパイルし,実行した結果が図\ref*{fig:2-1}である.
左側がclient.cを,右側をserver.cを実行した結果を表している.
server側で入力した文字列をclient側で同時に出力することができた.
また,imamuraと入力することで終了させることもできた.

なお,server2.cにおける\texttt{scanf("\%s", buf);}に対して\texttt{scanf("\%s\textbackslash n",buf);}
のように,\textbackslash を加えると入力したものが遅れて出力される現象が確認できた.
\lstinputlisting[caption={server2.c}\label{cd:2-1}]{code/server2.c}
\lstinputlisting[caption={client2.c}\label{cd:2-2}]{code/client2.c}


%server2.cとclient2.cの実行結果
\begin{figure}[H]
  \begin{center}
    \includegraphics*[scale=0.7]{figure/2-1.png}
  \end{center}
  \caption{server2.cとclient2.cの実行結果}
  \label{fig:2-1}
\end{figure}

\newpage
\section*{演習3}
作成したserver3.cとclient3.cはソースコード\ref*{cd:3-1},\ref*{cd:3-2}である.
このプログラムをコンパイルし,実行した結果が図\ref*{fig:3-1}である.
左側がclient3.cを,右側をserver3.cを実行した結果である.
server側ではclient側で設定したIPアドレス等を出力することができたし,client側ではserver側で設定したIPアドレス等を出力することができた.
\lstinputlisting[caption={server3.c}\label{cd:3-1}]{code/server3.c}
\lstinputlisting[caption={client3.c}\label{cd:3-2}]{code/client3.c}


%server3.cとclient3.cの実行結果
\begin{figure}[H]
  \begin{center}
    \includegraphics*[scale=0.7]{figure/3-1.png}
  \end{center}
  \caption{server3.cとclient3.cの実行結果}
  \label{fig:3-1}
\end{figure}

\newpage
\section*{演習4}
作成したserver4.cとclient4.cはソースコード\ref*{cd:4-1},\ref*{cd:4-2}である.
このプログラムをコンパイルし,実行した結果が図\ref*{fig:4-1}である.
左側と真ん中がclient4.cを,右側をserver4.cを実行した結果である.
左側と真ん中は同時を多少時間を開けて実行したとしても,0から10まで出力されるのが確認できた.
\lstinputlisting[caption={server4.c}\label{cd:4-2}]{code/server4.c}
\lstinputlisting[caption={client4.c}\label{cd:4-1}]{code/client4.c}

%server4.cとclient4.cの実行結果
\begin{figure}[H]
  \begin{center}
    \includegraphics*[scale=0.55]{figure/4-1.png}
  \end{center}
  \caption{server4.cとclient4.cの実行結果}
  \label{fig:4-1}
\end{figure}


\section*{備考}
今回作成したプログラムをgithub上に公開した.\\
\url{https://github.com/tanusai646/network_program/tree/main/4}
\end{document}