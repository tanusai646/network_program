\documentclass[documentclass]{jsarticle}
\usepackage[top=25truemm,bottom=25truemm,left=20truemm,right=20truemm]{geometry}
\usepackage{listings, jlisting, color}
\usepackage[dvipdfmx]{graphicx}
\usepackage{pdfpages}
\usepackage{amsmath}
\usepackage{amssymb, latexsym}
\usepackage{mathtools}
\usepackage{multirow}
\usepackage{color}
\usepackage{ulem}
\usepackage{here}
\usepackage{wrapfig}
\usepackage{tikz}
% 使用する関数の宣言
\usepackage{url}
\usepackage{hyperref}
\hypersetup{
    pdfauthor={222C1021 今村優希},
}
\usetikzlibrary{intersections, calc, arrows, positioning, arrows.meta}

\newcommand{\Add}[1]{\textcolor{red}{#1}}
\newcommand{\Erase}[1]{\textcolor{red}{\sout{\textcolor{black}{#1}}}}
\newcommand{\ctext}[1]{\raise0.2ex\hbox{\textcircled{\scriptsize{#1}}}}

\lstset{
  basicstyle=\ttfamily,
  breaklines=true,
  frame=single,
  tabsize=3,
}

\begin{document}
\title{ネットワークプログラミング\\ 第8回演習レポート}
\author{222C1021 今村優希}
\maketitle

\newpage

\section*{演習2}
作成したhtmlファイルはソースコード1である.
\lstinputlisting[caption={index.html}]{code/index.html}

\newpage

\section*{演習3}
上記のファイルをブラウザを用いて表示させた結果が図である.
\begin{figure}[H]
  \begin{center}
    \includegraphics*[scale=0.5]{figure/1-1.png}
  \end{center}
  \caption{ブラウザの画面}
  \label{fig:3-1}
\end{figure}

\section*{備考}

\end{document}